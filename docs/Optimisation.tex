
%%%
%%% CHAPTER
%%%
\chapter{Global Optimisation Methods}\label{Chapter:GlobalOpt}


%%%
%%% Section
%%%
\section{Introduction}\label{Chapter:GlobalOpt:Section:Introduction}

%%% Subsection
\subsection{Main Definitions}\label{Chapter:GlobalOpt:Section:Definitions}
The main aim of optimisation methods is to find the {\it global optimum} (maximum or minimum) of a function. However, global optimum are often difficult to obtain except in a few problems. 

A \underline{constrained minimisation problem} is defined as,
\begin{shaded}
   Given an {\it objective function} $f\left(\cdot\right):\mathcal{S}\subset\mathcal{R}^{n}\rightarrow\mathcal{R}$, where $\mathcal{S}$ is either the constraints or the solutions set of the optimisation problem $\mathcal{P}$,
\begin{displaymath}
   \mathcal{P}: \text{ Minimise }\; f(x)\; \text{ such that }\; x\in\mathcal{S}, 
\end{displaymath}
where $x=\left(x_{1},\cdots,x_{n}\right)$.
\end{shaded}

However, if $\mathcal{S}=\mathcal{R}^{n}$ then the problem is called  \underline{non-constrained minimisation problem}, i.e., the objective function $f$ must be minimised without explicit constraints,
\begin{shaded}
   Given an {\it objective function} $f\left(\cdot\right):\mathcal{S}\subseteq\mathcal{R}^{n}\rightarrow\mathcal{R}$, where $\mathcal{S}$ is either the constraints or the solutions set of the optimisation problem $\mathcal{P}$.
\begin{displaymath}
    \mathcal{P}: \text{ Minimise }\; f(x)\; \text{ such that }\; x\in\mathcal{R}^{n}. 
\end{displaymath}
where $x=\left(x_{1},\cdots,x_{n}\right)$.
\end{shaded}
Indeed, for such problems we need to obtain first- and second-order necessary and sufficient conditions for optimality, defined by the following theorems\footnote{Proof of these theorems and definitions can be found in any advanced linear algebra and optimisation text books.}:

%%% Theorem:
\begin{thm}[First-Order Optimality Conditions]\label{Chapter:GlobalOpt:TheoremFirstOrder}
Let $f:\mathcal{R}^{n}\rightarrow\mathcal{R}$ be differentiable at a solution coordinate $x^{\ast}\in\mathcal{R}^{n}$. If $x^{\ast}$ is a local solution of problem $\mathcal{P}$, then $\nabla f\left(x^{\ast}\right)=0$.
\end{thm}

%%% Theorem:
\begin{thm}[Second-Order Optimality Conditions]\label{Chapter:GlobalOpt:TheoremSecondOrder}
Let $f:\mathcal{R}^{n}\rightarrow\mathcal{R}$ be twice differentiable at a solution coordinate $x^{\ast}\in\mathcal{R}^{n}$. 
   \begin{enumerate}
       \item If $x^{\ast}$ is a local solution of problem $\mathcal{P}$, then $\nabla f\left(x^{\ast}\right)=0$ and $\nabla^{2}f\left(x^{\ast}\right)$ is positive semi-definite (necessary condition).
       \item If $\nabla f\left(x^{\ast}\right)=0$ and $\nabla^{2}f\left(x^{\ast}\right)$ is positive definite, then there is an $\alpha > 0$ such that $f\left(x\right)\ge f\left(x^{\ast}\right)+\alpha\left\|x+\alpha\right\|^{2}$ for all $x$ near $x^{\ast}$ (sufficient condition).
   \end{enumerate}
\end{thm}
Theorem~\ref{Chapter:GlobalOpt:TheoremSecondOrder} establishes both 2$^{\text{nd}}$-order necessary and sufficient conditions for a problem whereas Theorem~\ref{Chapter:GlobalOpt:TheoremFirstOrder} describes 1$^{\text{st}}$-order necessary conditions. The sufficiency conditions is given by the following definitions:

%%% Definition:
\begin{defn}[Convex Sets and Functions] For
   \begin{enumerate}
       \item A subset $\mathcal{C}\subset\mathcal{R}^{n}$ is assumed convex if for every pair $x$ and $y$ taken from $\mathcal{C}$, the entire line segment connecting $x$ and $y$ is also contained in $\mathcal{C}$,
           \begin{displaymath}
              \left[x,\right]\subset\mathcal{C}\;\;\;\text{ where }\;\;\; \left[x,y\right]=\left\{\left(1-\lambda\right)x +\lambda y : 0\le\lambda\le 1\right\}
           \end{displaymath}
       \item A function $f:\mathcal{R}^{n}\rightarrow \mathcal{R}\cup\left\{\pm\infty\right\}$ is assumed convex if the set
           \begin{displaymath}
              \text{epi}(f) = \left\{\left(\mu,x\right): f(x)\le \mu\right\}
           \end{displaymath}
           is a convex subset of $\mathcal{R}^{n+1}$. We can also define the set 
           \begin{displaymath}
              \text{dom}(f) = \left\{x\in\mathcal{R}^{n} : f(x) < +\infty\right\}
           \end{displaymath}
   \end{enumerate}
\end{defn}

%%% Lemma:
\begin{lem}
    The function $f : \mathcal{R}^{n}\rightarrow\mathcal{R}$ is convex \underline{if and only if} for every two solution coordinates, $x_{1},x_{2}\in\text{ dom}(f)$ and $\lambda\in\left[0,1\right]$ there is
       \begin{displaymath}
          f\left(\lambda x_{1} +\left(1-\lambda\right)x_{2}\right) \le \lambda f\left(x_{1}\right) + \left(1 - \lambda\right)f\left(x_{2}\right),
       \end{displaymath}
i.e., the secant line connecting $\left(x_{1},f\left(x_{1}\right)\right)$ and $\left(x_{2},f\left(x_{2}\right)\right)$ lies above the graph of $f$.
\end{lem}


%%% Theorem:
\begin{thm}\label{Theorem_Convex1}
   Let $f:\mathcal{R}^{n}\rightarrow\mathcal{R}\cup\left\{\pm\infty\right\}$ be convex. If $x^{\ast}\in\text{dom}(f)$ is a local solution to problem $\mathcal{P}$, then $x^{\ast}$ is a global solution to the problem $\mathcal{P}$.
\end{thm}

If $f$ is a differentiable convex function then a better solution can be obtained, thus this solution can be obtained from the following lemma.
%%% Lemma:
\begin{lem}\label{Lemma_Convex}
    Let $f : \mathcal{R}^{n}\rightarrow\mathcal{R}\cup\left\{+\infty\right\}$ be convex:
       \begin{enumerate}
          \item Given  $x\in\text{ dom}(f)$ and $d\in\mathcal{R}^{n}$, the ratio
             \begin{displaymath}
                 \frc{f(x+td)-f(x)}{t}
             \end{displaymath}
             is a monotonically increasing function of $t\in\left(0,+\infty\right)$.
          \item For every $x\in\text{ dom}(f)$ and $d\in\mathcal{R}^{n}$, the directive derivative $f^{\prime}\left(x;d\right)$ always exist and is given by
             \begin{displaymath}
                 f^{\prime}\left(x;d\right) = \displaystyle\inf_{t>0} \frc{f(x+td)-f(x)}{t}.
             \end{displaymath}
          \item For every $x\in\text{ dom}(f)$, the function $f^{\prime}\left(x;\cdot\right)$ is positively homogeneous,
             \begin{displaymath}
                 f^{\prime}\left(x;\alpha d\right) = \alpha f^{\prime}\left(x;d\right) \hspace{.5cm} \forall d\in\mathcal{R}^{n},\; \alpha > 0,
             \end{displaymath}
              and subadditive,
             \begin{displaymath}
                 f^{\prime}\left(x;u+v\right) \le f^{\prime}\left(x;u\right) + f^{\prime}\left(x;v\right).
             \end{displaymath}
       \end{enumerate}
\end{lem}
From {\it Lemma}~\ref{Lemma_Convex},
%%% Theorem:
\begin{thm}\label{Theorem_Convex2}
   Let $f:\mathcal{R}^{n}\rightarrow\mathcal{R}\cup\left\{+\infty\right\}$ be convex and also let's suppose that $x^{\ast}\in\mathcal{R}^{n}$ is a solution coordinate at which $f$ is differentiable. Then $x^{\ast}$ is a global solution to the problem $\mathcal{P}$ if and only if $\nabla f\left(x^{\ast}\right)=0$.
\end{thm}

%%% Theorem:
\begin{thm}\label{Theorem_Convex3}
   Let $f:\mathcal{R}^{n}\rightarrow\mathcal{R}$:
     \begin{enumerate}
        \item If $f$ is differentiable on $\mathcal{R}^{n}$, then the following statements are equivalent:
             \begin{enumerate}
                 \item $f$ is convex;
                 \item $f(y)\ge f(x)+\nabla f(x)^{\text{T}}\left(y-x\right)$ for all $[x,y]\in\mathcal{R}^{n}$;
                 \item $\left(\nabla f(x)-\nabla f(y)\right)^{\text{T}}(x-y) \ge 0$ for all $[x,y]\in\mathcal{R}^{n}$;
             \end{enumerate}
        \item If $f$ is twice differentiable then$f$ is convex if and only if $f$ is positive semi-definite for all $x\in\mathcal{R}^{n}$.
     \end{enumerate}
\end{thm}

This establishes that $f^{\prime}\left(x;d\right)$ exists for all $x\in\text{ dom}(f)$ and $d\in\mathcal{R}^{n}$. For a brief discussion of the continuity properties of this function is available in the following lemma:
%%% Lemma:
\begin{lem}
    Let $f:\mathcal{R}^{n}\rightarrow\mathcal{R}\cup\left\{+\infty\right\}$ be convex. Then, $f$ is bounded in a neighbourhood of a solution coordinate $x^{\ast}$ if and only if $f$ is Lipschitz\footnote{
Let $f:\mathcal{R}^{n}\rightarrow\mathcal{R}^{n}$, the Lipschitz-continuous function is defined by:
    \begin{enumerate}[(a)]
       \item Given an open set $\mathcal{B}\subseteq\mathcal{R}^{n}$, a function $f$ is called Lipschitz-continuous on the open subset $\mathcal{B}$ is there is a constant $\Lambda\in\mathcal{R}_{0}^{+}$ such that,
           \begin{displaymath}
               \left\|f(x)-f(y)\right\| \le  \Lambda\left\|x-y\right\|, \hspace{.5cm} \forall[x,y]\in\mathcal{B}.
           \end{displaymath}
       \item The function $f$ is called locally Lipschitz-continuous, if for each $z\in\mathcal{R}^{n}$ there is an $L>0$ such that $f$ is Lipschitz-continuous on the open sphere of center $z$ and radius $L$,
           \begin{displaymath}
              \mathcal{B}_{L}(z) = \left\{y\in\mathcal{R}^{n} :\left\|y-z\right\| < L\right\}.
           \end{displaymath}
       \item If $f$ is Lipschitz-continuous on all $\mathcal{R}^{n}$ space $\left(\text{i.e., }\mathcal{B}=\mathcal{R}^{n}\text{ in (a) above}\right)$, then $f$ is called globally Lipschitz-continuous.

    \end{enumerate}}
 in a neighbourhood of $x^{\ast}$.
\end{lem}

%%% Definition:
\begin{defn}[Local Minima]
   A solution coordinate $x^{\ast}$ is a local solution of a problem if there is a neighbourhood 
   \begin{displaymath}
           \Omega\left(x^{\ast},\Omega\right) = \left\{ z \left|\;\;\left|z-x^{\ast}\right| < \delta ; \delta > 0\right.\right\},
   \end{displaymath}
   such that $f\left(x^{\ast}\right)\leq f\left(x\right)\;\;\forall x\in\Omega\left(x^{\ast},\Omega\right) \cap \mathcal{S}$.
\end{defn}

%%% Definition:
\begin{defn}[Global Minima]
   A solution coordinate $x^{\ast}\in\mathcal{S}$ is a global solution of a problem if $f\left(x^{\ast}\right)\leq f\left(x\right)\;\;\forall x\in\mathcal{S}$.
\end{defn}

\bigskip
Optimisation methods are often based on iterative methods, in which after an initial solution $\left(\text{or estimate}, x^{\circ}\right)$ is chosen, a sequence of points $\left\{x^{k}\right\}$ is obtained through a prescribed algorithm. The sequence must converge to the solution $x^{\ast}$ of the problem.
%%% Definition:
\begin{defn}[Local Covergence]
   An algorithm is locally convergent if there is a positive scalar $\sigma$ such that for any initial solution coordinate $x^{\circ}\in\mathcal{R}^{n}$ $\left(\text{or } x^{\circ}\in\mathcal{S}\right)$, with $\left|x^{\circ}-x^{\ast}\right|\leq\sigma$, the algorithm generates a sequence of solutions that converges to the solution $x^{\ast}$.
\end{defn}

%%% Definition:
\begin{defn}[Global Covergence]
   An iterative algorithm is globally convergent if for any initial solution coordinate $x^{\circ}\in\mathcal{R}^{n}$ $\left(\text{or } x^{\circ}\in\mathcal{S}\right)$, the algorithm generates a sequence of solutions that converges to the solution $x^{\ast}$.
\end{defn}

\bigskip
Convergence is assumed asymptotic if the solution is not found after a finite number of iterations. Generally, globally convergent algorithms change search directions at each new coordinate, $x^{k}$. Therefore,

%%% Definition:
\begin{defn}[Descendant Direction] A linearly independent vector $d\in\mathcal{R}^{n}$ represents a descendant direction of a real function $f$ in $x\in\mathcal{R}^{n}$ if there is $\delta>0$, such that $f\left(x+t d\right)<f\left(x\right)$ for any $t\in\left.\left(0,\delta\right.\right]$.   
\end{defn}

%%% Definition:
\begin{defn}[Viable Direction] A linearly independent vector $d\in\mathcal{R}^{n}$ represents a viable direction of a real function $f$ in $x\in\mathcal{S}$ if there is $\theta >0$, such that $x+t d\in \mathcal{S}$ for all $t\in\left.\left(0,\theta\right.\right]$.   
\end{defn}

%%% Subsection
\subsection{Conditions of Optimality}\label{Chapter:GlobalOpt:Section:ConditionsOptimality}
Let $f:\mathcal{R}^{n}\rightarrow\mathcal{R}$ with $x^{\ast}$ be a global minimum of the function.

%%% Subsection
\subsubsection{Non-Constrained Optimisation}



%%% Subsection
\subsubsection{Constrained Optimisation}






%%%
%%% Section
%%%
\section{Introduction to Newton-based Methods}\label{Chapter:GlobalOpt:Section:NewtonMethods}

%%% Section
\subsection{Linear Search}\label{Chapter:GlobalOpt:Section:NewtonMethods:LinearSearch}

\begin{shaded}
    Let's consider an open domain $\mathcal{C}$ in $\mathcal{R}^{n}$ defined as
\begin{displaymath}
    \mathcal{C} = \left\{x\in\mathcal{R}^{n}; 0<x_{i},1, \forall i=1,\cdots,n\right\}
\end{displaymath}
and a non-linear function $F$ differentiable in $\mathcal{C}$. If this function is twice differentiable in the open space $\mathcal{D}\subset\mathcal{R}^{n}$ and assuming an 
\end{shaded}

%%%
%%% Section
%%%
\section{Legendre Polynomials}

\clearpage
%%%
%%% Section
%%%
\section{Simulated Annealing Algorithm}\label{Chapter:GlobalOpt:Section:SimulatedAnnealing}

\begin{algorithm}[H]\scriptsize
   \SetKwData{Left}{left}\SetKwData{This}{this}\SetKwData{Up}{up}
   \SetKwFunction{Union}{Union}\SetKwFunction{FindCompress}{FindCompress}
   \SetKwInOut{Input}{Input}\SetKwInOut{Output}{Output}\SetKwInOut{Calculate}{Calculate}\SetKwInOut{Set}{Set}\SetKwInOut{Adjust}{Adjust}

      \Input{Given an initial estimative of the array of unknowns $\underline{X}$ and the cooling schedule list:}
      \Calculate{$F\left[\underline{X}^{\left(0\right)}\right]$}
      \Set{ $\underline{X}^{\left(\text{opt}\right)} = \underline{X}^{\left(0\right)}$ and $F\left[\underline{X}^{\left(\text{opt}\right)}\right] = F\left[\underline{X}^{\left(0\right)}\right]$ }

      \While{ Convergence }{ 
         \For{$l \leftarrow 1$ \KwTo$N_{T}$}{
             \For{$m \leftarrow 1$ \KwTo$N_{S}$}{
                 \For{$h \leftarrow 1$ \KwTo$N$}{
                     Update of array $\underline{X}$ \;
                 
                     \If{$\left[\underline{X}^{\left(i+1,h\right)}\ge\underline{a}\right.$ {\bf or} $\left.\underline{X}^{\left(i+1,h\right)}\le\underline{b}\right]$}{
                           \Calculate{$F\left[\underline{X}^{\left(i+1,h\right)}\right]$} 
                        }
                 
                     \eIf{$\left\{F\left[\underline{X}^{\left(i+1,h\right)}\right] > F\left[\underline{X}^{\left(i,h\right)}\right]\right\}$}{
                           \If{ (Metropolis criteria is accepted) }{
                                $\underline{X}^{(i,h)} = \underline{X}^{(i+1,h)}$\;
                                $F\left[\underline{X}^{(i,h)}\right] = F\left[\underline{X}^{(i+1,h)}\right]$      
                              }
                         }    {
                                $\underline{X}^{(i,h)} = \underline{X}^{(i+1,h)}$
                              }
                 
                     \If{ $F\left[\underline{X}^{(i+1,h)}\right] < F\left[\underline{X}^{(\text{opt})}\right]$ }{
                                $\underline{X}^{(\text{opt})} = \underline{X}^{(i+1,h)}$\;
                                $F\left[\underline{X}^{(\text{opt})}\right] = F\left[\underline{X}^{(i+1,h)}\right]$    
                        }
%
                 } % Loop h
             } % Loop m
%
             \Adjust{ Step-size matrix}
%
         } % Loop l
%
         \eIf{ $\left|F\left[\underline{X}^{(i,h)}\right] - F\left[\underline{X}^{(i+1,h)}\right]\right| < \epsilon$ }{
               \Calculate{ $F\left[\underline{X}^{(\text{opt})}\right]$}
             }{
               \Adjust{ Temperature parameter }} 
     } % While 



 \caption{Simmulated Annealing Algorithm.}
\end{algorithm}



\clearpage


%%%
%%% Section
%%%
\section{Particle Swarm Algorithm}
