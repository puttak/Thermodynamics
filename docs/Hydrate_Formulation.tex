
%%%
%%% CHAPTER
%%%
\chapter{Hydrate Formation}\label{Chapter:Hydrate}


%%%
%%% Section
%%%
\section{Mass Conservation and Degrees of Freedom Analysis}\label{Chapter:Hydrate:Section:MassConservation}

Assuming an isothermal flash with the following phases present in the equilibrium:
  \begin{center}
    \begin{tabular}{l l l l}
      $L$: & liquid phase & $H$: & hydrate phase \\
      $V$: & vapour phase & $I$: & ice phase \\ 
    \end{tabular}
  \end{center}
$L$, $V$, $H$ and $I$ are the molar fraction of each phase. In addition, $F$ represents the molar fraction of the feeding stream. Finally, let's define $x_{i}^{j}$ as the molar fraction of component $i \left(=1,2,\cdots,n_{c}\right)$ in phase $j \left(=L,V,H,I,F\right)$. Thus, assuming a closed system (\ie no mass transfer across a defined domain border):
  \begin{eqnarray}
     && L + V + H + I = 1 \\\label{Chapter:Hydrate:Eqn:Balance:MassFractionPhase}
     && \mfr[x]{i}{F}F = \mfr[x]{i}{V}V + \mfr[x]{i}{L}L + \mfr[x]{i}{H}H + \mfr[x]{i}{I}I, \label{Chapter:Hydrate:Eqn:Balance:MassFractionComponent}
  \end{eqnarray}
with the following linear constraints:
  \begin{equation}
    \summation_{i}^{n_{c}} \mfr[x]{i}{j} = 1, \hspace{2cm} j=L,V,H,I,F\label{Chapter:Hydrate:Eqn:Balance:MassFractionConstraint}
  \end{equation}
 
The resulting system of non-linear algebraic equations have $4n_{c}+6$ unknowns, \ie $\mfr[x]{i}{V}, \mfr[x]{i}{L}, \mfr[x]{i}{H}, \mfr[x]{i}{I}$ (each of them with $n_{c}$ components), $L$, $V$, $I$, $H$, temperature ($T$) and pressure ($P$).  In the case of methane hydrate, $n_{c}=2$ $\left(\text{\ie} CH_{4}+H_{2}O\right)$. Additionally, the ice phase ($I$) contains just water -- $w_{w}^{I}=1$, and the number of unknowns and equations (linear constraint for ice) are reduced.

In the equilibrium, the equality of phase fugacities for a given component $i$,
   \begin{equation}
      \mfr[f]{i}{V} = \mfr[f]{i}{L} = \mfr[f]{i}{H} = \mfr[f]{i}{I} \hspace{2cm}\text{with } i=1,\cdots,n_{c} \label{Chapter:Hydrate:Eqn:FugacityEquality}
   \end{equation}
Note that one of the relations in Eqn.~\ref{Chapter:Hydrate:Eqn:FugacityEquality} is not independent. Thus the total number of independent equations is $4n_{c}+4$, and the number of degrees of freedom $\left(\varphi\right)$, \ie   
   \begin{center} 
      (number of unknowns) - (number of independent equations),
   \end{center}
is equal to $2$.

If we specify the pair $\left[T,P\right]$, we can determine the composition of each component, $\mfr[x]{i}{j}$, and each phase $j$. The Gibb's phase rule,
  \begin{eqnarray}
     \varphi &=& 2 + n_{c} - n_{p} \\ \nonumber
             &=& 2 + 2 - n_{p} = 4 - n_{p},
  \end{eqnarray}
states the possibility of a quadruple point $\left(\varphi=0\right)$, however if we want to specify $\left[T,P\right]$ then a two-phase system may occur.



%%%
%%% Section
%%%
\section{Fugacity Models}\label{Chapter:Hydrate:Section:FugacityModels}

%%%
\subsection{Models for Liquid and Vapour Phases}\label{Chapter:Hydrate:Section:FugacityModels:Section:L_V}

%%%
\subsubsection{Soave-Redlich-Kwong Equation of State (SRK-EOS)}\label{Chapter:Hydrate:Section:FugacityModels:Section:L_V:Section:SRKEOS}
The cubic polynomial in $z$ (compressibility factor) representation of the SRK-EOS can be expressed as,
   \begin{equation}
      z^{3} - z^{2} + \left(A-B-B^{2}\right)z - AB = 0 \label{Chapter:Hydrate:Section:FugacityModels:Section:L_V:Section:SRKEOS:Eqn:SRKEOS}
   \end{equation}
where $A$ and $B$ are defined in Eqn.~\ref{Chapter:EOSPR:Section:PR:Eqn:CubicPR_Z_Parameters} with
   \begin{displaymath}
      a = \sum\limits_{i=1}^{n_{c}}\sum\limits_{j=1}^{n_{c}}x_{i}x_{j}a_{i,j} \hspace{1cm}\text{ and }\hspace{1cm} b = \sum\limits_{i=1}^{n_{c}}x_{i}b_{i}
   \end{displaymath}
with
   \begin{displaymath}
      a_{i}=0.42747\frc{\left(RT_{c,i}\right)^{2}}{P_{c,i}}, \hspace{1cm} b_{i}=0.08664\frc{RT_{c,i}}{P_{c,i}}\hspace{1cm}\text{ and }\hspace{1cm}a_{i,j} = \sqrt{\alpha_{i}a_{i}\alpha_{j}a_{j}}
   \end{displaymath}

The fugacity is  expressed as 
   \begin{eqnarray}
      f_{i}(T,P) &=& P\exp{\left[\left(z-1\right)-\ln{\left(z-B\right)}-\frc{A}{B}\ln{\left(1+\frc{B}{Z}\right)}\right]}  \\
      f\left(T,P,x_{i}\right) &=& x_{i}P\exp{\left[\frc{b_{i}}{b}\left(z-1\right)-\ln{\left(z-B\right)}-\frc{A}{B}\left(\frc{2\sum\limits_{j=1}^{n_{c}}x_{j}a_{i,j}}{a}-\frc{b_{i}}{b}\right)\ln{\left(1+\frc{B}{Z}\right)}\right]} 
   \end{eqnarray}

%%%
\subsubsection{Peng-Robinson Equation of State (PR-EOS)}\label{Chapter:Hydrate:Section:FugacityModels:Section:L_V:Section:PREOS}
The cubic polynomial in $z$ (compressibility factor) representation of the SRK-EOS can be expressed as,
   \begin{displaymath}
      z^{3} -\left(1-B\right)z^{2} + \left(A-2B-3B^{2}\right)z^{2} - \left(AB -B^{3} - B^{2}\right) = 0
   \end{displaymath}
(see Eqn.~\ref{Chapter:EOSPR:Section:PR:Eqn:CubicPR_Z}) with fugacities,
   \begin{eqnarray}
      f_{i}(T,P) &=& P\exp{\left\{\left(z-1\right)-\ln{\left(z-B\right)}-\frc{A}{2\sqrt{2}B}\ln{\left[\frc{z + \left(\sqrt{2}+1\right)}{z-\left(\sqrt{2}-1\right)}\right]}\right\}}  \\
      f\left(T,P,x_{i}\right) &=& x_{i}P\exp\left\{\frc{b_{i}}{b}\left(z-1\right)-\ln{\left(z-B\right)}-\frc{A}{1\sqrt{2}B}\left(\frc{2\sum\limits_{j=1}^{n_{c}}x_{j}a_{i,j}}{a}-\frc{b_{i}}{b}\right) \times \right.\nonumber \\
                              && \hspace{6cm}\left.\ln{\left[\frc{z + \left(1-\sqrt{2}\right)B}{z +\left(1+\sqrt{2}\right)B}\right]}\right\} 
   \end{eqnarray}

%%%
\subsection{Host-Component in the Liquid Phase}\label{Chapter:Hydrate:Section:FugacityModels:Section:HostLiquid}
The host component (hydrocarbon) in the liquid phase is constant for a given temperature according to the Henry's law for the solubility calculation of hydrocarbons in water -- negligible at low temperatures bur relatively high for high temperatures,
   \begin{equation}
      \mfr[x]{i}{L} = \mfr[f]{i}{L}\left[\mathcal{H}_{i}\exp{\left(\frc{P\overline{V}^{\infty}}{RT}\right)}\right]^{-1},\hspace{1cm} i=1,2,\cdots,n_{c}
   \end{equation}
where $\overline{V}^{\infty}$ is the partial molar volume at infinite dilution, $\mathcal{H}_{i}$ is the Henry's constant~\citep{Prausnitz_2000} defined as
   \begin{equation}
      -\ln{\left[\frc{\mathcal{H}_{i}}{101325}\right]} = \mfr[\mathcal{H}]{i}{1} + \frc{\mfr[\mathcal{H}]{i}{2}}{T} + \mfr[\mathcal{H}]{i}{3}\ln{T} +  \mfr[\mathcal{H}]{i}{4}T,
   \end{equation}
where $\mfr[\mathcal{H}]{i}{j}$ are constants. For binary systems, the molar fraction of water in the liquid phase can be calculated as,
   \begin{equation}
     \mfr[x]{w}{L} = 1 - \mfr[f]{i}{L}\left[\mathcal{H}_{i}\exp{\left(\frc{P\overline{V}^{\infty}}{RT}\right)}\right].
   \end{equation}




%%%
%%% Section
%%%
\section{Models for the Hydrate Phase}\label{Chapter:Hydrate:Section:HydrateModels}


%%%
\subsection{van der Waals and Platteeuw Model}\label{Chapter:Hydrate:Section:HydrateModels:Section:vdW_Platteeuw}
This model~\citep{vdWP_1959} is based on the assumption of the equality of the chemical potential of water in the hydrate phase ($H$) and a second phase, either liquid ($L$) or ice ($I$). We can generalise (for simplicity in the notation) these two phases as $\Pi$:
   \begin{equation}
      \Delta\mfr[\mu]{w}{H} = \mfr[\mu]{w}{\beta} - \mfr[\mu]{w}{H} =  \mfr[\mu]{w}{\beta} - \mfr[\mu]{w}{\Pi} = \Delta\mfr[\mu]{w}{\Pi},
   \end{equation}
where $\mfr[\mu]{w}{\beta}$ is the chemical potential of water when the clathrate cage is {\it empty} -- this is often considered as a meta-stable phase as the hydrate structure can still `welcome' a guest-molecule (hydrocarbon). And,
   \begin{equation}
     \mfr[\mu]{w}{H}(T,P) = \mfr[\mu]{w}{\beta}(T,P) + KT\summation_{m}\nu_{m}\ln{\left[1-\summation_{i}\Theta_{im}(T,P)\right]}
   \end{equation}

%%%
\subsection{Fugacity of Water}\label{Chapter:Hydrate:Section:HydrateModels:Section:FugacityWater}
\citet{klauda_2000}:
   \begin{equation}
      \mfr[f]{w}{H}(T,P) = \mfr[f]{w}{\beta}(T,P)\exp{\left[-\frc{\Delta\mfr[\mu]{w}{H}(T,P)}{RT}\right]},
   \end{equation}
with 
   \begin{eqnarray}
       \Delta\mfr[\mu]{w}{H} = -RT\summation_{m}\nu_{}\ln{\left(1-\summation_{i}\Theta_{im}\right)}\hspace{2cm}\text{ and}, \\
       \mfr[f]{w}{\beta}(T,P) = \mfr[P]{w}{\text{sat},\beta}(T)\exp{\left\{\frc{\mfr[v]{w}{\beta}\left[P-\mfr[P]{w}{\text{sat},\beta}(T)\right]}{RT}\right\}},
   \end{eqnarray}
where $\mfr[P]{w}{\text{sat},\beta}$ is the vapour pressure of the water in the hydrate phase, $\mfr[v]{w}{\beta}$ is the volume of the empty hydrate that depends on the hydrate structure, $k$ is the Boltzmann constant, $\nu_{m}$ is the number of cavities (\ie cages) od the hydrate structure of type $m$ per water molecule, $\Theta_{im}(T,P)$ is the volume fraction occupied by the guest molecule $i$ in cavity $m$ and is calculated from the Langmuir adsorption isothermal relation,
   \begin{equation}
      \Theta_{im}(t,P) = \frc{C_{im}(T)f_{i}(T,P)}{1=\summation_{m}C_{m}(T)f_{i}(T,P)}
   \end{equation}
