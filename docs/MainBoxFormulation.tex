
%%%
%%% CHAPTER
%%%
\chapter{Thermodynamic Formulation}\label{Chapter:ThermodynamicFormulation}
 
%%% SECTION
\section{Mass Balance}\label{Chapter:ThermodynamicFormulation:Section:MassBalance}
For a system with $n_{p}$ phases and $n_{c}$ components, the mass of each $j$ phase is,
\begin{equation}
m^{(j)} = \summation_{i=1}^{n_{c}}\mfr[m]{i}{j}\;\;\;\text{ for }j=1,2,\cdots,n_{p}
\label{Eqn_MassBalance_Phase}
\end{equation}

And the mass of each component $i\left(=1,2,\cdots,n_{c}\right)$,
\begin{equation}
m_{i} = m_{i}^{(1)} + m_{i}^{(2)} + \cdots + m_{i}^{\left(n_{p}\right)} = \summation_{j=1}^{n_{p}}\mfr[m]{i}{j}
\label{Eqn_MassBalance_Component}
\end{equation}

The total mass $m$ can be define as,
\begin{eqnarray}
m & = & m^{(1)} + m^{(2)} + \cdots + m^{\left(n_{p}\right)} = \summation_{j=1}^{n_{p}}m^{(j)} \nonumber \\
  & = & m_{1} + m_{2} + \cdots + m_{n_{c}} = \summation_{i=1}^{n_{c}}m_{i} \nonumber 
\label{Eqn_MassBalance_Mass}
\end{eqnarray}

Defining the intensive properties:
\begin{eqnarray}
&& \mfr[w]{i}{j} = \frc{\mfr[m]{i}{j}}{m^{(j)}}, \hspace{2cm} z_{i} = \frc{m_{i}}{m},\label{Eqn_MassBalance_MassFraction} \\
&& \Pi^{(j)} = \frc{m^{(j)}}{m}\label{Eqn_MassBalance_PhaseMassFraction} 
\end{eqnarray}
where $\mfr[w]{i}{j}$ is the mass fraction of component $i$ in phase $j$, $\Pi^{j}$ is the mass fraction of phase $j$ and $z_{i}$ is the overall feed mass fraction of component $i$. From Eqns.~\ref{Eqn_MassBalance_Mass}-~\ref{Eqn_MassBalance_PhaseMassFraction}, we can define normalised quantities, i.e., fractions, and we can add the following constraints:
\begin{eqnarray}
&&\mfr[w]{n_{c}}{j} = 1 - \summation_{i=1}^{n_{c}-1}\mfr[w]{i}{j},\hspace{1cm}  z_{n_{c}} = 1 - \summation_{i=1}^{n_{c}-1}z_{i},  \label{Eqn_MassBalance_MassFraction2} \\
&&\Pi^{\left(n_{p}\right)} = 1 - \left(\Pi^{(1)} + \Pi^{(2)} + \cdots + \Pi^{\left(n_{p}-1\right)}\right) = 1 - \summation_{j=1}^{n_{p}-1} \Pi^{j}\hspace{1cm} j=1,2,\cdots,n_{p} \label{Eqn_MassBalance_PhaseMassFraction2}
\end{eqnarray} 
with
\begin{displaymath}
  z_{i} = \frc{m_{i}}{m} = \frc{\summation_{j=1}^{n_{p}}\mfr[m]{i}{j}}{m} \frc{m^{(j)}}{m^{(j)}} = \underbrace{\frc{m^{(j)}}{m}}_{\red{\Pi^{j}}} \cdot \overbrace{\frc{\summation_{j=1}^{n_{p}}\mfr[m]{i}{j}}{m^{(j)}}}^{\red{\summation_{j=1}^{n_{p}}\mfr[w]{i}{j}}} 
\end{displaymath}
leading to
\begin{equation}
z_{i} = \Pi^{(1)}w_{i}^{(1)} + \Pi^{(2)}w_{i}^{(2)} + \cdots + \Pi^{\left(n_{p}\right)}w_{i}^{\left(n_{p}\right)}
\label{Eqn_MassBalance_FeedMassFractionConstraint}
\end{equation}
with
\begin{equation}
0\leq\Pi^{(j)}\leq 1 \hspace{3cm} 0\leq \mfr[w]{i}{j}\leq 1
\end{equation}

If the solution is contained within \red{$n_{p}$} phases, the inequality can be rewritten as,
\begin{equation} 
0 < \Pi^{(j)} < 1
\end{equation}
Thus
\begin{displaymath}
\Pi^{(k)} = 1 - \summation_{j=1,j\neq k}^{n_{p}} \Pi^{(j)} \neq 0
\end{displaymath}
Therefore from Eqn.~\ref{Eqn_MassBalance_FeedMassFractionConstraint},
\begin{equation}
w_{i}^{(k)} = \frc{z_{i}-\summation_{j=1,j\neq k}^{n_{p}}\Pi^{j}\mfr[w]{i}{j}}{\Pi^{(k)}} = \frc{z_{i} - \summation_{j=1,j\neq k}^{n_{p}}\Pi^{j}\mfr[w]{i}{j}}{1-\summation_{j=1,j\neq k}^{n_{p}}\Pi^{(j)}}
\label{Eqn_MassBalance_FeedMassFractionConstraint2}
\end{equation}


\begin{shaded}\noindent
   For 2 phases, $\Pi^{(1)}=L$ and $\Pi^{(2)}=V$, Eqn.~\ref{Eqn_MassBalance_FeedMassFractionConstraint} becomes:
     \begin{displaymath}
      \mfr[w]{i}{V} = \frc{z_{i}-L\mfr[w]{i}{L}}{1-L}, \hspace{1cm}\text{ for } i=1,2,\cdots,n_{c}
     \end{displaymath}
     And for 3 phases, $\Pi^{(1)}=L$, $\Pi^{(2)}=V$ and $\Pi^{(3)}=H$ (see Section~\ref{Chapter:Hydrate:Section:MassConservation}):
        \begin{displaymath}
           \mfr[w]{i}{H} = \frc{z_{i}-\left(V \mfr[w]{i}{V} + L \mfr[w]{i}{L}\right)}{1 - \left(V \mfr[w]{i}{V} + L \mfr[w]{i}{L}\right)}
        \end{displaymath}
\end{shaded}

%% SECTION
\section{Free Gibbs Energy}\label{Chapter:ThermodynamicFormulation:Section:GibbsEnergy}
For equilibrium problems (e.g., vapour-liquid equilibrium, VLE), the total free Gibbs energy can be expressed as,
\begin{equation}
G = \summation_{j=1}^{n_{p}}\left(\summation_{i=1}^{n_{c}} \mfr[m]{i}{j}\mfr[\mu]{i}{j}\right)
\label{MassGibbs_Definition}
\end{equation} 
this represents
\begin{displaymath}
G = \summation_{i=1}^{n_{c}} \mfr[m]{i}{1}\mfr[\mu]{i}{1} + \summation_{i=1}^{n_{c}} \mfr[m]{i}{2}\mfr[\mu]{i}{2} + \cdots + \summation_{i=1}^{n_{c}} \mfr[m]{i}{n_{p}}\mfr[\mu]{i}{n_{p}}
\end{displaymath}
The chemical potential $\left(\mu\right)$ can be written as a function of the mass of each component at each phase,
\begin{eqnarray}
\mfr[\mu]{i}{j} &=& \mfr[\mu]{i}{j}\left(\mfr[m]{1}{j}, \mfr[m]{2}{j}, \cdots, \mfr[m]{n_{c}}{j}\right) = \mfr[\mu]{i}{j}\left(\frc{\mfr[m]{1}{j}}{\mfr[m]{}{j}}, \frc{\mfr[m]{2}{j}}{\mfr[m]{}{j}}, \cdots, \frc{\mfr[m]{n_{c}}{j}}{\mfr[m]{}{j}}\right)\nonumber \\
&=& \mfr[\mu]{i}{j}\left(\mfr[w]{1}{j}, \mfr[w]{2}{j}, \cdots, \mfr[w]{n_{c}}{j}\right)\label{chempotential_functional}
\end{eqnarray}
Dividing Eqn.~\ref{MassGibbs_Definition} by $m$,
\begin{equation}
g = \frc{G}{m} = \summation_{j=1}^{n_{p}}\left(\summation_{i=1}^{n_{c}}\frc{\mfr[m]{i}{j}}{m}\mfr[\mu]{i}{j}\right) = \summation_{i=1}^{n_{c}}\frc{\mfr[m]{i}{1}}{m^{(1)}} \cdot \frc{m^{(1)}}{m} \mfr[\mu]{i}{1} + \cdots + \summation_{i=1}^{n_{c}}\frc{\mfr[m]{i}{n_{p}}}{\mfr[m]{}{n_{p}}} \cdot \frc{\mfr[m]{}{n_{p}}}{m} \mfr[\mu]{i}{n_{p}}
\label{MassGibbs_Definition2}
\end{equation}
And we can rewrite Eqn.~\ref{MassGibbs_Definition2} as,
\begin{equation}
g = \summation_{i=1}^{n_{c}}\Pi^{(1)}\mfr[w]{i}{1}\mfr[\mu]{i}{1} + \summation_{i=1}^{n_{c}}\Pi^{(2)}\mfr[w]{i}{2}\mfr[\mu]{i}{2} + \cdots + \summation_{i=1}^{n_{c}}\mfr[\Pi]{}{n_{p}}\mfr[w]{i}{n_{p}}\mfr[\mu]{i}{n_{p}} = \summation_{j=1}^{n_{p}}\left[\summation_{i}^{n_{c}}\mfr[\Pi]{}{j}\mfr[w]{i}{j}\mfr[\mu]{i}{j}\right] \label{MassGibbs_Definition3}
\end{equation}


%\begin{mdframed}[style=JFrame]
\begin{shaded}\noindent
   For 2 ($L$ and $V$) phases,
     \begin{displaymath}
        g = \summation_{i=1}^{n_{c}}L\mfr[w]{i}{L}\mfr[\mu]{i}{L} + \summation_{i=1}^{n_{c}}V\mfr[w]{i}{V}\mfr[\mu]{i}{V}
     \end{displaymath}
     and 3 ($L$, $V$ and $H$) phases,
     \begin{displaymath}
       g = \summation_{i=1}^{n_{c}}L\mfr[w]{i}{L}\mfr[\mu]{i}{L} + \summation_{i=1}^{n_{c}}V\mfr[w]{i}{V}\mfr[\mu]{i}{V} + \summation_{i=1}^{n_{c}}H\mfr[w]{i}{H}\mfr[\mu]{i}{H}
     \end{displaymath}
\end{shaded}
%\end{mdframed}


Now replacing Eqn.~\ref{Eqn_MassBalance_MassFraction2} in Eqn.~\ref{MassGibbs_Definition3},
\begin{eqnarray}
g &=& \summation_{i=1}^{n_{c}-1}\Pi^{(1)}\mfr[w]{i}{1}\left(\mfr[\mu]{i}{1}-\mfr[\mu]{n_{c}}{1}\right) + \Pi^{(1)}\mfr[\mu]{n_{c}}{1} + \summation_{i=1}^{n_{c}-1}\Pi^{(2)}\mfr[w]{i}{2}\left(\mfr[\mu]{i}{2}-\mfr[\mu]{n_{c}}{2}\right) + \Pi^{(2)}\mfr[\mu]{n_{c}}{2} + \cdots + \nonumber \\
  && \summation_{i=1}^{n_{c}-1}\mfr[\Pi]{}{n_{p}}\mfr[w]{i}{n_{p}}\left(\mfr[\mu]{i}{n_{p}}-\mfr[\mu]{n_{c}}{n_{p}}\right) + \mfr[\Pi]{}{n_{p}}\mfr[\mu]{n_{c}}{n_{p}}
\label{MassGibbs_Definition4}
\end{eqnarray}
Now replacing Eqns.~\ref{Eqn_MassBalance_PhaseMassFraction2} and~\ref{Eqn_MassBalance_FeedMassFractionConstraint2} in~\ref{MassGibbs_Definition4}:
\begin{eqnarray}
g &=& \summation_{i=1}^{n_{c}}\Pi^{(1)}\mfr[w]{i}{1}\left(\mfr[\mu]{i}{1}-\mfr[\mu]{n_{c}}{1}\right) + \Pi^{(1)}\mfr[\mu]{n_{c}}{1} + \cdots + \summation_{i=1}^{n_{c}}\mfr[\Pi]{}{n_{p}-1}\mfr[w]{i}{n_{p}-1}\left(\mfr[\mu]{i}{n_{p}-1}-\mfr[\mu]{n_{c}}{n_{p}-1}\right) +  \nonumber \\
   && \mfr[\Pi]{}{n_{p}-1}\mfr[\mu]{n_{c}}{n_{p}-1} + \summation_{i=1}^{n_{c}-1}\left[z_{i} - \summation_{j=1}^{n_{p}-1}\mfr[\Pi]{}{j}\mfr[w]{i}{j}\right]\left(\mfr[\mu]{i}{n_{p}}-\mfr[\mu]{n_{c}}{n_{p}}\right) + \mfr[\Pi]{}{n_{p}}\mfr[\mu]{n_{c}}{n_{p}}\nonumber
\end{eqnarray}

Rearranging
\begin{eqnarray}
g &=& \summation_{i=1}^{n_{c}}\mfr[\Pi]{}{1}\mfr[w]{i}{1}\left[\left(\mfr[\mu]{i}{1}-\mfr[\mu]{i}{n_{p}}\right) - \left(\mfr[\mu]{n_{c}}{1}-\mfr[\mu]{n_{c}}{n_{p}}\right)\right] + \cdots +  
\nonumber \\
&& \summation_{i=1}^{n_{c}}\mfr[\Pi]{}{n_{p}-1}\mfr[w]{i}{n_{p}-1}\left[\left(\mfr[\mu]{i}{n_{p}-1}-\mfr[\mu]{i}{n_{p}}\right) - \left(\mfr[\mu]{n_{c}}{n_{p}-1}-\mfr[\mu]{n_{c}}{n_{p}}\right)\right] + \summation_{j=1}^{n_{p}-1}\Pi^{(j)}\mfr[\mu]{n_{c}}{j} + \nonumber \\
&& \summation_{i=1}^{n_{c}-1}z_{i}\left(\mfr[\mu]{i}{n_{p}}-\mfr[\mu]{n_{c}}{n_{p}}\right) +\left(1-\summation_{j=1}^{n_{p}-1}\Pi^{(j)}\right)\mfr[\mu]{n_{c}}{n_{p}} \nonumber
\end{eqnarray}

With a compressed notation:

\begin{eqnarray}
g &=& \summation_{i=1}^{n_{c}-1}\left\{\summation_{j=1}^{n_{p}-1}\Pi^{(j)}\mfr[w]{i}{j}\left[\left(\mfr[\mu]{i}{j}-\mfr[\mu]{i}{n_{p}}\right) - \left(\mfr[\mu]{n_{c}}{j}-\mfr[\mu]{n_{c}}{n_{p}}\right)\right]\right\} + \summation_{j=1}^{n_{p}-1}\Pi^{(j)}\left(\mfr[\mu]{n_{c}}{j}-\mfr[\mu]{n_{c}}{n_{p}}\right) +  \nonumber \\
&& \hspace{3cm} \summation_{i=1}^{n_{c}-1} z_{i}\mfr[\mu]{i}{n_{p}} + \underbrace{\left(1-\summation_{i=1}^{n_{c}-1}z_{i}\right)}_{\red{z_{n_{c}}}}\mfr[\mu]{n_{c}}{n_{p}} \nonumber 
\end{eqnarray}
Thus
\red{\begin{eqnarray}
g &=& \summation_{i=1}^{n_{c}-1}\left\{\summation_{j=1}^{n_{p}-1}\Pi^{j}\mfr[w]{i}{j}\left[\left(\mfr[\mu]{i}{j}-\mfr[\mu]{i}{n_{p}}\right) - \left(\mfr[\mu]{n_{c}}{j}-\mfr[\mu]{n_{c}}{n_{p}}\right)\right]\right\} + \nonumber \\
       && \summation_{j=1}^{n_{p}-1}\Pi^{(j)}\left(\mfr[\mu]{n_{c}}{j}-\mfr[\mu]{n_{c}}{n_{p}}\right) + \summation_{i=1}^{n_{c}}z_{i}\mfr[\mu]{i}{n_{p}}
\end{eqnarray}}

The chemical potential was defined as a functional (Eqn.~\ref{chempotential_functional}):
\begin{eqnarray}
&& \mfr[\mu]{i}{1} = \mfr[\mu]{i}{1}\left(\mfr[w]{1}{1},\mfr[w]{2}{1},\cdots,\mfr[w]{n_{c}}{1}\right) \nonumber \\
&& \mfr[\mu]{i}{2} = \mfr[\mu]{i}{2}\left(\mfr[w]{1}{2},\mfr[w]{2}{2},\cdots,\mfr[w]{n_{c}}{2}\right) \nonumber \\
&& \hspace{2cm}\vdots \nonumber \\
&& \mfr[\mu]{i}{n_{p}} = \mfr[\mu]{i}{n_{p}}\left(\mfr[w]{1}{n_{p}},\mfr[w]{2}{n_{p}},\cdots,\mfr[w]{n_{c}}{n_{p}}\right) \label{ChemPotDef2}
\end{eqnarray}
with constraint Eqn.~\ref{Eqn_MassBalance_FeedMassFractionConstraint2} in Eqn.~\ref{ChemPotDef2},
\begin{eqnarray}
\mfr[\mu]{l}{k} &=& \mfr[\mu]{l}{k}\left(\mfr[w]{i}{j},\Pi^{(j)}\right) \nonumber \\
           &=& \mfr[\mu]{l}{k}\left(\mfr[w]{1}{1},\cdots,\mfr[w]{n_{c}-1}{1},\mfr[w]{1}{2},\cdots,\mfr[w]{n_{c}-1}{2},\cdots,\mfr[w]{1}{n_{p}},\mfr[w]{2}{n_{p}},\cdots,\mfr[w]{n_{c}-1}{n_{p}},\Pi^{(1)},\cdots,\mfr[\Pi]{}{n_{p}}\right)\nonumber
\end{eqnarray}
with $i=1,2,\cdots,l,\cdots,n_{c}-1$ and $j = 1,2,\cdots,n_{p}$ and $j\neq k$


%%% Section
\section{Box Formulation}\label{Chapter:ThermodynamicFormulation:Section:BoxFormulation}
The mass-based free Gibbs energy may be expressed a function of the mass fraction of each component at each phase, $\mfr[w]{i}{j}$ and the mass fraction of each individual phase, $\mfr[\Pi]{}{j}$, in equilibrium conditions,
\begin{displaymath}
   g = g\left(\mfr[w]{i}{j},\mfr[\Pi]{}{j}\right).
\end{displaymath}
From classical thermodynamic~\citep{Balmer_Book}, $\mfr[\mu]{i}{j}$ can be defined as a function of the fugacity coefficient, $\varphi_{i}^{j}$,
\begin{equation}
   \mfr[\mu]{i}{j}= R T\left[\ln{\mfr[\varphi]{i}{j}} + \ln{\left(P \mfr[x]{i}{j}\right)}\right] + \theta_{i}\left(T\right),\label{Chapter:ThermodynamicFormulation:Section:BoxFormulation:Eqn:ChemPot}
\end{equation}
where $\theta_{i}$ is an integration constant~\citep{SmithVanNess_Book}. The fugacity coefficient is defined as,
\begin{equation}
   \mfr[\varphi]{i}{j} = \frc{\mfr[f]{i}{j}}{P \mfr[x]{i}{j}},\label{Chapter:ThermodynamicFormulation:Section:BoxFormulation:Eqn:FugacityCoeff}
\end{equation}
where $f$ and $x$ are the fugacity and molar fraction, respectively. The fugacity coefficient can be calculated for each component at each phase through an equation of state. In the term $\ln{\left(P\mfr[x]{i}{j}\right)}$ of Eqn.~\ref{Chapter:ThermodynamicFormulation:Section:BoxFormulation:Eqn:ChemPot}, the species are present in all phases. The mole and mass fractions $\left(\mfr[x]{i}{j}\text{ and }\mfr[w]{i}{j}\right)$ must not be equal to zero, otherwise this term would tend to infinity. Thus, the constraint
  \begin{displaymath}
    \summation_{i=1}^{n_{c}} \mfr[x]{i}{j} = 1 = \summation_{i=1}^{n_{c}} \mfr[w]{i}{j},
  \end{displaymath}
enforces that mass fractions must not be equal to unity, otherwise all other compositions would be zero. Therefore
   \begin{displaymath}
        0 < \mfr[w]{i}{j} < 1.
   \end{displaymath}

%%% Section
\section{Minimum Free Gibbs Energy}\label{Chapter:ThermodynamicFormulation:Section:MinimumGibbsEnergy}
In multiphase systems at constant pressure and temperature conditions, the concentration of all $n_{c}$ species in equilibrium are unknown. The minimum energy principle states that at equilibrium, the extensive parameters result in the minimum Gibbs free energy at constant pressure and temperature~\citep{Callen_Book}. Therefore,  

\begin{shaded}
   \begin{center}
     {\bf Statement of the VLE Problem}
   \end{center}

   Given $T$, $P$ and $z_{i}$ $\left(i=1,2,\cdots,n_{c}\right)$. Find the set of $\mfr[w]{1}{L},\cdots,\mfr[w]{n_{c}-1}{L}$ and $L$ that minimises
   \begin{eqnarray}
      g\left(\mfr[w]{1}{L},\cdots,\mfr[w]{n_{c}-1}{L}, L\right) &=& \summation_{i=1}^{n_{c}-1}L\mfr[w]{i}{L}\left[\left(\mfr[\mu]{i}{L}-\mfr[\mu]{i}{V}\right)-\left(\mfr[\mu]{n_{c}}{L}-\mfr[\mu]{n_{c}}{V}\right)\right] \nonumber\\
         && + L\left(\mfr[\mu]{i}{L}-\mfr[\mu]{i}{V}\right) + \summation_{i}^{n_{c}}z_{i}\mfr[\mu]{i}{V} 
   \end{eqnarray}
   with the constraints:
\[ 
\left\{
  \begin{tabular}{l}
  $0 < L < 1$ \\
  $0 < \mfr[w]{i}{L} < 1$, \hspace{1cm} $i=1,2,\cdots,n_{c}-1$\\
  $\summation_{i=1}^{n_{c}}\mfr[w]{i}{L} < 1$\\ 
  \end{tabular}
\right.
\]
where,
\begin{displaymath}
   \mfr[\mu]{i}{L}=\mfr[\mu]{i}{L}\left(\mfr[w]{1}{L},\cdots,\mfr[w]{n_{c}-1}{L}\right)\text{ and }\mfr[\mu]{i}{V}=\mfr[\mu]{i}{V}\left(\mfr[w]{1}{L},\cdots,\mfr[w]{n_{c}-1}{L}, L\right)
\end{displaymath}

\end{shaded}
